\section{Технологии конструирования ПО}

{\em Технология конструирования программного обеспечения (ТКПО)}~--- 
система инженерных принципов для создания экономичного ПО, 
которое надежно и эффективно работает в реальных компьютерах.

Тематика предметной области ТКПО носит актуальный характер 
в современных условиях, однако, основные проблемы ТКПО 
сформировались достаточно давно и человечество накопило 
значительный опыт их решения и выработало различные методы 
решения задач, связанных с разработкой программных систем.
В период 80--90-х годов XX века успехи микроэлектроники 
привели к резкому увеличению производительности компьютеров 
и значительному снижению их стоимости, вследствие чего 
на передний план вышли задачи совершенствования качества 
компьютерных приложений, возможности которых целиком 
определяются программным обеспечением. Чрезвычайно актуальными 
стали следующие проблемы:

\begin{itemize}
  \item аппаратная сложность опережает умение строить ПО, 
использующее потенциальные возможности аппаратуры;
  \item умение строить новые программы отстает от требований 
к новым программам;
  \item возможностям эксплуатирования существующих программ 
угрожает низкое качество их разработки.
\end{itemize}

Ключом к решению этих проблем является грамотная организация процесса 
создания ПО, реализация технологических принципов промышленного 
конструирования программных систем.

%Методы канструирования ПО должны обеспечивать 
%эффективное решение следующих задач:

%\begin{itemize}
%  \item планирование и оценка проекта;
%  \item анализ системных и программных требований;
%  \item проектирование алгоритмов, структур данных и программных структур;
%  \item кодирование;
%  \item тестирование;
%  \item сопровождение.
%\end{itemize}


\subsection{Классический жизненый цикл}
Наиболее популярной парадигмой ТКПО является {\em классический 
жизненый цикл}, который описывает последовательность шагов 
конструирования программного обеспечения, обеспечиающую 
систематический, упорядоченный подход к промышленной 
разработке, использованию и сопровождению ПО.

В классическом жизненном цикле выделяются следующие этапы разработки 
программного обоспечения:
\begin{enumerate}
  \item Постановка задачи. Системный анализ и анализ требований. 
  \item Проектирование.
  \item Кодирование.
  \item Тестирование.
  \item Сопровождение.
\end{enumerate}

На этапе системного анализа выделяются компоненты системы 
и их взаимодействие, оцениваются необходимые ресурсы и 
начинается решение задачи планирования проекта ПО. 
Во время анализа требований уточняются и детализируются 
функции ПО, его характеристики и интерфейс, завершается 
решение задачи планирования проекта.

Проектирование состоит в создании представлений:
\begin{itemize}
  \item архитектуры ПО;
  \item модульной структуры ПО;
  \item алгоритмической структуры ПО;
  \item структуры данных;
  \item входного и выходного интерфейса.
\end{itemize}
При решении задач проектирования основное внимание 
уделяется качеству будущего программного продукта.

Кодирование состоит в переводе результатов проектирования 
в текст на языке программирования. 

Тестирование~--- выполнение программы для выявления дефектов в функциях, 
логике и форме реализации программного продукта.

Сопровождение~--- это внесение изменений в эксплуатируемое ПО. 
Цели изменений:
\begin{itemize}
  \item исправление ошибок;
  \item адаптация к изменениям внешней для ПО среды;
  \item усовершенствование ПО по требованиям заказчика.
\end{itemize}

\subsection{Модели разработки ПО}
Существует несколько различных моделей конструирования ПО, 
среди основных выделяют следующие модели:
\begin{itemize}
  \item {\em однопроходная}~--- линейная последовательность 
этапов конструирования;
  \item {\em итерационная(инкрементная)}~--- в начале процесса определяются 
все пользовательские и системные требования, оставшаяся 
часть конструирования выполняется в виде последовательности версий, 
первая версия реализует часть запланированных возможностей 
и базовый функционал, следующая версия реализует даполнительные 
возможности и характеристики и т.д., пока не будет получена 
полная система. 
  \item {\em спиральная}~--- реализует эволюционную стратегию 
конструирования ПО, то есть система строится в виде последовательности 
версий, но в начале процесса определены не все требования, которые 
уточняются в результате разработки версий, также перед началом 
каждой итерации разработки оценивается риск, и если он слишком велик, 
то проект может быть остановлен
  \item {\em компонентно-ориентированная}~--- является развитием 
эволюционной стратегии конструирования, особое внимание здесь уделяется 
созданию библиотек компонентов и повторному их использованию 
в программных системах.
\end{itemize}

\subsection{Тяжеловесные и облегченные процессы}
Традиционно для разработки ПО использовались строго упорядочивающие 
тяжеловесные (heavyweight) процессы. В этих процессах прогнозируется 
весь объем предстоящих работ, составляется системная спецификация, 
разрабатывается проектная документация, которая строго регламентирует 
непосредственную реализацию~--- кодирование. Данные процессы 
позволяют создавать большие и сложные программные системы, 
качественно организовать работу при многочисленной группе 
разработчиков разной квалификации. Подобные прогнозирующие 
процессы применяются при фиксированных требованиях, 
так как гибкое изменение программного продукта и подстройка 
под меняющиеся требования в подобных процессах затруднена.

В последние годы появилась группа новых, облегченных (lightweight) 
процессов, также называемых подвижными (agile). Данные процессы 
требуют меньшего объема документации и ориентированы на человека. 
В отличее от тяжеловесных процессов в них отсутствует бюракратизм. 
Облегченные процессы позволяют учитывать частые изменения требований 
заказчика к программному продукту. Однако реализация данных процессов 
возможна при малочисленной группе высококвалифицированных разрабочиков 
и заказчиках, готовых участвовать в разработке. 

Примерами облегченных процессов могут служить экстремальное 
программирование (eXtreme Programming, XP) и Scrum-методология. 
Данные процессы ориентированы на получение максимального результата 
в минимально возможные сроки. В своей методологии опираются на принципы, 
продиктованные здравым смыслом и позволяющие эффективно использовать 
ресурсы и повышать качество ПО.
  
\endinput
