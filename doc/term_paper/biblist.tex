\newpage
\renewcommand{\bibname}{Список литературы и интернет-ресурсов}
\begin{thebibliography}{2009}

\bibitem{trpobook}
C.A. Орлов.
{\em Технологии разработки программного обеспечения. Учебное пособие.}~--- СПб.:
ПИТЕР, 2003.~--- 480 с.



\bibitem{dsbbook}
Д.С. Бургонский.
{\em <<Системное программирование. Практикум.>>}~--- М.:
МГИУ, 2009.~--- 48 с.

\bibitem{schildtbook}
Г. Шилдт.
{\em Справочник программиста по C/C++}, 3-е изд. Ж Пер. с англ.~--- М.:
ООО \glqqИ.Д. Вильямс\grqq, 2006.~--- 432 с.

\bibitem{tcpipbook}
Й. Снейдер
{\em Эффективное программирование TCP/IP. Библиотека программиста.}~--- СПб.: Питер, 2002.~--- 320 с.

\bibitem{unixbook}
А.М. Робачевский.  
{\em Операционная система UNIX.}~--- СПб.: БХВ-Петербург, 2003.~--- 528 с.

\end{thebibliography}

\endinput
