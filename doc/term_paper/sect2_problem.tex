\section{Постановка задачи}
Для начала дадим определение калькулятора, разработке 
которого посвящена данная работа. Под калькулятором будем 
понимать компьютерную программу для выполнения операций над числами 
и алгебраическими формулами. Инженерные калькуляторы 
предназначены для научных и инженерных расчётов, 
имеют большое количество функций, включая вычисление 
всех элементарных функций, статистические расчёты и т.п.

Рассмотрим существующие программные решения в области 
вычислений и компьютерной алгебры. В настоящий момент 
существует огромное множество разных по функциональности 
программ, реализующих функции калькуляторов. 
Практически с каждой современной операционной системой 
поставляется системный калькулятор, имеющий некоторые 
возможности инженерных расчетов. Примерами могут служить 
следующие калькуляторы:
\begin{itemize}
\item {\bf Calculator} (calc.exe)~--- компонент Microsoft Windows, имитирующий работу обычного карманного калькулятора.
\item {\bf Kcalc}~--- калькулятор для рабочей среды KDE, работает в UNIX-подобных операционных системах, распространяется свободно на условиях GNU GPL.
\item {\bf Gcalctool}~--- калькулятор для рабочей среды GNOME, работает в UNIX-подобных операционных системах, имеет базовый, финансовый и научный режимы работы, распространяется свободно на условиях GNU GPL.
\item {\bf Calculator} (Mac OS X)~--- калькулятор разработаннный Apple Inc., 
встроен в Mac OS X, может иметь вид простого, инженерного и для программистов.
\end{itemize}
Данные калькуляторы как правило имеют несколько режимов: простой, инженерный,
финансовый, для программистов. Также эти программы обладают простым 
интерфейсом и ориентированны на небольшие числовые расчеты.

В UNIX-системах есть пакеты {\bf bc} и {\bf dc} для арифметических 
вычислений с произвольной точностью. Однако, данные программы имеют 
интерфейс командной строки и поэтому неудобны для обычных пользователей. 

На практике в научных расчетах используют специальные математические пакеты, 
предоставляющие обширные возможности в области вычислений и 
математического анализа. Наиболее известными являются системы:

\begin{itemize}
\item {\bf Mathcad}~--- программа для выполнения и документирования инженерных 
и научных расчётов, реализована для платформы MS Windows, является 
коммерческим ПО.

\item {\bf Mathematica}~--- система компьютерной алгебры компании Wolfram Research, 
поддерживаемые ОС: Microsoft Windows, Mac OS X, GNU/Linux, Sun Solaris, 
лицензия~--- коммерческая.

\item {\bf Maxima}~--- свободная система компьютерной алгебры, 
кроссплатформенное ПО, распространяется свободно на условиях GNU GPL.
\end{itemize}

Подобные системы сложны в освоении, для работы в них пользователям 
требуется получить дополнительное образование, использовать 
различные учебники и справочники. Большинство таких программных продуктов
являются проприетарными и имеют большую цену, поэтому недоступны 
рядовым пользователям.

В последнее время появились мощные интеллектуальные поисковые системы, 
способные выполнять различные математические расчеты и операции. 
Например такие функции предоставляют следующие ресурсы: 
\begin{itemize}
\item {\bf wolframalpha.com}~--- поисковая система, интеллектуальный 
<<вычислительный движок знаний>>, способный вычислять ответ 
на запрос пользователя, основываясь на собственной базе знаний; 
данная система способна переводить данные между различными 
единицами измерения, системами счисления, подбирать общую 
формулу последовательности, вычислять суммы, пределы, интегралы, 
решать уравнения и системы уравнений, производить операции с матрицами 
(в настоящее время не поддерживает русский язык).

\item {\bf nigma.ru}~--- российская интеллектуальная метапоисковая система, 
позволяет производить простейшие арифметические преобразования, 
решать математические задачи различной степени сложности, 
имеет возможности вывода хода решения.
\end{itemize}

\subsection{Формулировка требований}
Перечислим минимальные требования, предъявляемые к калькулятору.
Необходимо, чтобы калькулятор предоставлял пользователю 
возможность ввода математического выражения и затем вычислял 
введенное выражение. Например, пользователь может ввести 
выражение $23/3+45*(12-56)$ и после ввода должен получить 
результат его вычисления $-1972.333333333$.

Система должна уметь выполнять четыре  арифметических действия 
(сложение, вычитание, умножение, деление) над целыми 32-разрядными 
числами и числами с плавающей точкой. Также должна быть реализована 
возможность сохранения значений в переменные для их дальнейщего 
использования (например, возможность определить переменную 
$X$ равную $69.1582$ и далее использовать ее в выражениях). 
Калькулятор должен иметь удобный и понятный пользовательский интерфейс.

Также желательно, чтобы в калькуляторе были реализованы 
дополнительные возможности, которые могут понадобится пользователям.
Данные возможности условно можно разделить на две группы.
Первые относяться к инженерным функциям калькулятора:

\begin{enumerate}
  \item Поддержка математических функций (см. табл.~\ref{tab:1}).
    
\begin{table}
\caption{Стандартные математические функции, 
используемые в инженерных калькуляторах.}\label{tab:1}
\renewcommand{\arraystretch}{1.3}
\begin{tabular}{|c|l|}
 \hline
 \multicolumn{2}{|c|}{Степенные функции}
 \\ \hline
 $1/x$    & обратное число
 \\ \hline
 $x^2$    & квадрат числа
 \\ \hline
 $x^3$    & куб числа
 \\ \hline
 $x^y$    & возведение в степень
 \\ \hline
 $\sqrt{x}$& квадратный корень
 \\ \hline
 $\sqrt[3]{x}$ & кубический корень
 \\ \hline
 
 \multicolumn{2}{|c|}{Тригонометрические функции}
 \\ \hline
 $\sin(x)$ & синус угла
 \\ \hline
 $\cos(x)$ & косинус угла
 \\ \hline
 $\tg(x)$  & тангенс угла
 \\ \hline
 
 \multicolumn{2}{|c|}{Обратные тригономертические функции}
 \\ \hline
 $\arcsin(x)$ & арксинус
 \\ \hline
 $\arccos(x)$ & арккосинус
 \\ \hline
 $\arctg(x)$  & арктангенс
 \\ \hline
 
 \multicolumn{2}{|c|}{Гиперболические функции}
 \\ \hline
 $\sh(x)$ & гиперболический синус
 \\ \hline
 $\ch(x)$ & гиперболический косинус
 \\ \hline
 $\th(x)$ & гиперболический тангенс
 \\ \hline
 
 \multicolumn{2}{|c|}{Экспоненты}
 \\ \hline
 $e^x$  & экспонента
 \\ \hline
 $10^x$ & десятичная экспонента
 \\ \hline
       
 \multicolumn{2}{|c|}{Логарифмы}  
 \\ \hline
 $\ln(x)$    & натуральный логарифм
 \\ \hline
 $\lg(x)$    & десятичный логарифм
 \\ \hline
 $\log_2(x)$ & двоичный логарифм
 \\ \hline
 $\log_y(x)$ & логарифм по произвольному основанию
 \\ \hline
      
 \multicolumn{2}{|c|}{Прочие функции} 
 \\ \hline  
 $\myabs(x)$  & абсолютое значение, модуль
 \\ \hline
 $\myint(x)$  & целая часть
 \\ \hline
 $\myfrac(x)$ & дробная часть
 \\ \hline
 $\myrand(x)$ & псевдослучайное число
 \\ \hline
\end{tabular}
\end{table}

  \item Поддержка различных способов представления чисел (двоичная,
        восьмеричная, десятичная, шестнадцатиричная системы счисления)
        и углов (градусы, грады, радианы).

  \item Поддержка побитовых операций (not~--- отрицание, 
        or~--- или, and~--- и, xor~--- исключающее или).

  \item Поддержка расчетов по комбинаторным формулам (размещения, сочетания,
        перестановки).

  \item Вычисление погрешностей результатов по заданным погрешностям входных
        значений.
\end{enumerate}

Ко второй группе относятся специальные математические 
возможности калькулятора:
\begin{enumerate}
  \item Дополнительные математические объекты.
  \begin{enumerate}
     \item Поддержка арифметики многократной точности
           (поддержка длинных целых чисел).

     \item Поддержка рациональных чисел.

     \item Поддержка длинных вещественных чисел.

     \item Поддержка комплексных чисел и основных операций с ними.

     \item Поддержка n-мерных векторов и
           основных операций с ними.

     \item Поддержка матриц и основных операций с ними.

     \item Поддержка полиномов и основных операций с ними.
  \end{enumerate}


  \item Поддержка решения стандартных задач.
  \begin{enumerate}
     \item Факторизация чисел.

     \item Нахождение НОД и НОК.

     \item Решение линейных систем уравнений.

     \item Решение квадратных уравнений.

     \item Дифференцирование элементарных функций.

     \item Решение линейных систем неравенств.

     \item Построение 2D графиков.

     \item Построение 3D графиков.
  \end{enumerate}

  \item Поддержка численных алгоритмов решения задач.
  \begin{enumerate}
     \item Нахождение корней уравнений.

     \item Решение систем линейных уравнений.

     \item Приближение функций.

     \item Численное дифференцирование.

     \item Численное интегрирование.
  \end{enumerate}

\end{enumerate}

\endinput
